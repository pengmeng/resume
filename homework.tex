%%%%%%%%%%%%%%%%%%%%%%%%%%%%%%%%%%%%%%%%%
% Short Sectioned Assignment
% LaTeX Template
% Version 1.0 (5/5/12)
%
% This template has been downloaded from:
% http://www.LaTeXTemplates.com
%
% Original author:
% Frits Wenneker (http://www.howtotex.com)
%
% License:
% CC BY-NC-SA 3.0 (http://creativecommons.org/licenses/by-nc-sa/3.0/)
%
%%%%%%%%%%%%%%%%%%%%%%%%%%%%%%%%%%%%%%%%%

%----------------------------------------------------------------------------------------
%	PACKAGES AND OTHER DOCUMENT CONFIGURATIONS
%----------------------------------------------------------------------------------------

\documentclass[letterpaper, 11pt]{scrartcl} % A4 paper and 11pt font size

\usepackage[T1]{fontenc} % Use 8-bit encoding that has 256 glyphs
\usepackage{fourier} % Use the Adobe Utopia font for the document - comment this line to return to the LaTeX default
\usepackage[english]{babel} % English language/hyphenation
\usepackage{amsmath,amsfonts,amsthm} % Math packages

%Chinese support
\usepackage{fontspec,xltxtra,xunicode}
\setromanfont{SimSun}
\XeTeXlinebreaklocale ``zh''
\XeTeXlinebreakskip = 0pt plus 1pt minus 0.1pt

\usepackage{sectsty} % Allows customizing section commands
\allsectionsfont{\normalfont\scshape} % Make all sections centered, the default font and small caps

\usepackage[textheight=8in,footskip=90pt]{geometry}
\usepackage{fancyhdr} % Custom headers and footers
\pagestyle{fancyplain} % Makes all pages in the document conform to the custom headers and footers
\fancyhead{} % No page header - if you want one, create it in the same way as the footers below
\fancyfoot[L]{} % Empty left footer
\fancyfoot[R]{} % Empty center footer
\fancyfoot[C]{\thepage} % Page numbering for right footer
\renewcommand{\headrulewidth}{0pt} % Remove header underlines
\renewcommand{\footrulewidth}{0pt} % Remove footer underlines
\setlength{\headheight}{0pt} % Customize the height of the header

\numberwithin{equation}{section} % Number equations within sections (i.e. 1.1, 1.2, 2.1, 2.2 instead of 1, 2, 3, 4)
\numberwithin{figure}{section} % Number figures within sections (i.e. 1.1, 1.2, 2.1, 2.2 instead of 1, 2, 3, 4)
\numberwithin{table}{section} % Number tables within sections (i.e. 1.1, 1.2, 2.1, 2.2 instead of 1, 2, 3, 4)

\setlength\parindent{0pt} % Removes all indentation from paragraphs - comment this line for an assignment with lots of text

%----------------------------------------------------------------------------------------
%	TITLE SECTION
%----------------------------------------------------------------------------------------

\newcommand{\horrule}[1]{\rule{\linewidth}{#1}} % Create horizontal rule command with 1 argument of height

\title{	
\normalfont \normalsize 
\textsc{\UTF{6D4B}\UTF{8BD5}??} \\ [10pt] % Your university, school and/or department name(s)
\horrule{0.5pt} \\[0.2cm] % Thin top horizontal rule
\huge EECS 499 : Homework 4 \\ % The assignment title
\horrule{0.5pt} \\[0.2cm] % Thick bottom horizontal rule
}

\author{Peng Meng} % Your name

\date{\normalsize\today} % Today's date or a custom date

\begin{document}

\maketitle % Print the title

%----------------------------------------------------------------------------------------
%	PROBLEM 1
%----------------------------------------------------------------------------------------

\section{Complete the calculations}
\begin{tabbing}
sample \hspace{4em} \= sample \hspace{2em} \= sample \hspace{2em} \= sample \hspace{4em} \= sample \hspace{2em} \= sample \kill
Rule \> Support \> P(lhs) \> Confidence \> P(hrs) \> Lift \\
\rule{36em}{0.5pt} \\
If M then P \> 0.25 \> 0.45 \> 0.556 \> 0.425 \> 1.31 \\
If P then M \> 0.25 \> 0.425 \> 0.588 \> 0.45 \> 1.31 \\
If M then C \> 0.20 \> 0.45 \>  0.444 \> 0.40 \> 1.11 \\
If C then M \> 0.20 \> 0.40 \> 0.500 \> 0.45 \> 1.11 \\
If P then C \> 0.15 \> 0.425 \> 0.353 \> 0.40 \> 0.88 \\
If C then P \> 0.15 \> 0.40 \> 0.375 \>0.425 \> 0.88\\
If (M, P) then C \> 0.05 \> 0.25 \> 0.20 \> 0.40 \> 0.5 \\
If (M, C) then P \> 0.05 \> 0.20 \> 0.25 \> 0.425 \> 0.5\\
\end{tabbing}

%\section{Lists}

%------------------------------------------------

%\subsection{Example of list (3*itemize)}
%\begin{itemize}
%	\item First item in a list 
%		\begin{itemize}
%		\item First item in a list 
%			\begin{itemize}
%			\item First item in a list 
%			\item Second item in a list 
%			\end{itemize}
%		\item Second item in a list 
%		\end{itemize}
%	\item Second item in a list 
%\end{itemize}

%------------------------------------------------

%\subsection{Example of list (enumerate)}
%\begin{enumerate}
%\item First item in a list 
%\item Second item in a list 
%\item Third item in a list
%\end{enumerate}

%----------------------------------------------------------------------------------------

\end{document}