%%%%%%%%%%%%%%%%%%%%%%%%%%%%%%%%%%%%%%%%%
% Beamer Presentation
% LaTeX Template
% Version 1.0 (10/11/12)
%
% This template has been downloaded from:
% http://www.LaTeXTemplates.com
%
% License:
% CC BY-NC-SA 3.0 (http://creativecommons.org/licenses/by-nc-sa/3.0/)
%
%%%%%%%%%%%%%%%%%%%%%%%%%%%%%%%%%%%%%%%%%

%----------------------------------------------------------------------------------------
%	PACKAGES AND THEMES
%----------------------------------------------------------------------------------------

\documentclass{beamer}

\mode<presentation> {

% The Beamer class comes with a number of default slide themes
% which change the colors and layouts of slides. Below this is a list
% of all the themes, uncomment each in turn to see what they look like.

%\usetheme{default}
%\usetheme{AnnArbor}
%\usetheme{Antibes}
%\usetheme{Bergen}
%\usetheme{Berkeley}
%\usetheme{Berlin}
%\usetheme{Boadilla}
%\usetheme{CambridgeUS}
%\usetheme{Copenhagen}
\usetheme{Darmstadt}
%\usetheme{Dresden}
%\usetheme{Frankfurt}
%\usetheme{Goettingen}
%\usetheme{Hannover}
%\usetheme{Ilmenau}
%\usetheme{JuanLesPins}
%\usetheme{Luebeck}
%\usetheme{Madrid}
%\usetheme{Malmoe}
%\usetheme{Marburg}
%\usetheme{Montpellier}
%\usetheme{PaloAlto}
%\usetheme{Pittsburgh}
%\usetheme{Rochester}
%\usetheme{Singapore}
%\usetheme{Szeged}
%\usetheme{Warsaw}

% As well as themes, the Beamer class has a number of color themes
% for any slide theme. Uncomment each of these in turn to see how it
% changes the colors of your current slide theme.

%\usecolortheme{albatross}
%\usecolortheme{beaver}
%\usecolortheme{beetle}
%\usecolortheme{crane}
%\usecolortheme{dolphin}
%\usecolortheme{dove}
%\usecolortheme{fly}
%\usecolortheme{lily}
%\usecolortheme{orchid}
%\usecolortheme{rose}
%\usecolortheme{seagull}
%\usecolortheme{seahorse}
%\usecolortheme{whale}
%\usecolortheme{wolverine}

%\setbeamertemplate{footline} % To remove the footer line in all slides uncomment this line
%\setbeamertemplate{footline}[page number] % To replace the footer line in all slides with a simple slide count uncomment this line

%\setbeamertemplate{navigation symbols}{} % To remove the navigation symbols from the bottom of all slides uncomment this line
}

\usepackage{graphicx} % Allows including images
\usepackage{booktabs} % Allows the use of \toprule, \midrule and \bottomrule in tables
\usepackage{listings} % For code listing
\DeclareGraphicsExtensions{.pdf,.png,.jpg}

%----------------------------------------------------------------------------------------
%	TITLE PAGE
%----------------------------------------------------------------------------------------

\title[Short title]{Fast Median Filter using CUDA} % The short title appears at the bottom of every slide, the full title is only on the title page

\author{Jian Shao, Peng Meng} % Your name
\institute[Northwestern U] % Your institution as it will appear on the bottom of every slide, may be shorthand to save space
{
Northwestern University \\ % Your institution for the title page
\medskip
%\textit{john@smith.com} % Your email address
}
\date{\today} % Date, can be changed to a custom date

\begin{document}

\begin{frame}
\titlepage % Print the title page as the first slide
\end{frame}

\begin{frame}
\frametitle{Overview} % Table of contents slide, comment this block out to remove it
\tableofcontents % Throughout your presentation, if you choose to use \section{} and \subsection{} commands, these will automatically be printed on this slide as an overview of your presentation
\end{frame}

%----------------------------------------------------------------------------------------
%	PRESENTATION SLIDES
%----------------------------------------------------------------------------------------

%------------------------------------------------
\section{Introduction} % Sections can be created in order to organize your presentation into discrete blocks, all sections and subsections are automatically printed in the table of contents as an overview of the talk
%------------------------------------------------

\subsection{Image Noise} % A subsection can be created just before a set of slides with a common theme to further break down your presentation into chunks

\begin{frame}
\frametitle{What Is Image Noise}
\begin{columns}
        \begin{column}{.5\textwidth}
		Image noise is random variation of brightness or color information in images that not present in the imaged object and is usually an aspect of electronic noise. It can be produced by the sensor and circuity of a scanner or digital camera. Image noise is an undesirable by-product of image capture that adds spurious and extraneous information. Simply, image noise is something we do not want and need to be removed.
        \end{column}
        \begin{column}{.5\textwidth}%\raggedleft
		\begin{figure}
	           	\includegraphics[scale=0.4]{img/salt_and_pepper}
		\end{figure}
        \end{column}
\end{columns}
\end{frame}

\begin{frame}
\frametitle{How To Reduce Image Noise}
\begin{columns}
        \begin{column}{.5\textwidth}
		There are many kinds of image noise, for example Gaussian noise, Salt-and-Pepper noise, Shot noise and Speckle noise. Different image noise can be reduced by different algorithm.\\~\\
		In our project, we concern about speckle noise and salt-and-pepper noise that can be effectively reduced by Median Filter.
        \end{column}
        \begin{column}{.5\textwidth}%\raggedleft
		\begin{figure}
	           	\includegraphics[scale=0.4]{img/salt_and_pepper_after}
		\end{figure}
        \end{column}
\end{columns}
\end{frame}

\subsection{Median Filter} 

\begin{frame}
\frametitle{What Is Median Filter}
\begin{itemize}
	\item Median Filter is a nonlinear digital filtering technique, often used to remove noise.
	\item Median Filter often acts as a typical pre-processing step to improve the results of later processing, for example, edge detection on a image.
	\item The main idea of the median filter is to run through the signal entry by entry, replacing each entry with the median of neighbouring entries. The pattern of neighbors is called 'window' in median filter, which slides entry by entry over the entire signal.
	\item The idea is obvious for 1D signals. For 2D signals such as images, the window pattern will become complicated. 
\end{itemize}
\end{frame}

\begin{frame}[fragile] % Need to use the fragile option when verbatim is used in the slide
\frametitle{2D Median Filter Pseudo Code}
\begin{scriptsize}
\begin{lstlisting}[frame=single, language=Pascal, basewidth=0.45em]
allocate outputPixelValue[image width][image height]
   allocate window[window width * window height]
   edgex := (window width / 2) rounded down
   edgey := (window height / 2) rounded down
   for x from edgex to image width - edgex
       for y from edgey to image height - edgey
           i = 0
           for fx from 0 to window width
               for fy from 0 to window height
                   window[i] := inputPixelValue[x + fx - edgex][y + fy - edgey]
                   i := i + 1
           sort entries in window[]
           outputPixelValue[x][y] := window[window width * window height / 2]
\end{lstlisting}
\end{scriptsize}
\end{frame}

%------------------------------------------------

%\begin{frame}
%\frametitle{Blocks of Highlighted Text}
%\begin{block}{Block 1}
%Lorem ipsum dolor sit amet, consectetur adipiscing elit. Integer lectus nisl, ultricies in feugiat rutrum, porttitor sit amet augue. %Aliquam ut tortor mauris. Sed volutpat ante purus, quis accumsan dolor.
%\end{block}

%\begin{block}{Block 2}
%Pellentesque sed tellus purus. Class aptent taciti sociosqu ad litora torquent per conubia nostra, per inceptos himenaeos. %Vestibulum quis magna at risus dictum tempor eu vitae velit.
%\end{block}

%\begin{block}{Block 3}
%Suspendisse tincidunt sagittis gravida. Curabitur condimentum, enim sed venenatis rutrum, ipsum neque consectetur orci, sed %blandit justo nisi ac lacus.
%\end{block}
%\end{frame}

%------------------------------------------------
\section{What we did}
%------------------------------------------------

\begin{frame}
\frametitle{Table}
\begin{table}
\begin{tabular}{l l l}
\toprule
\textbf{Treatments} & \textbf{Response 1} & \textbf{Response 2}\\
\midrule
Treatment 1 & 0.0003262 & 0.562 \\
Treatment 2 & 0.0015681 & 0.910 \\
Treatment 3 & 0.0009271 & 0.296 \\
\bottomrule
\end{tabular}
\caption{Table caption}
\end{table}
\end{frame}

%------------------------------------------------

\begin{frame}
\frametitle{Theorem}
\begin{theorem}[Mass--energy equivalence]
$E = mc^2$
\end{theorem}
\end{frame}

%------------------------------------------------

\begin{frame}[fragile] % Need to use the fragile option when verbatim is used in the slide
\frametitle{Verbatim}
\begin{example}[Theorem Slide Code]
\begin{verbatim}
\begin{frame}
\frametitle{Theorem}
\begin{theorem}[Mass--energy equivalence]
$E = mc^2$
\end{theorem}
\end{frame}\end{verbatim}
\end{example}
\end{frame}

%------------------------------------------------

\begin{frame}
\frametitle{Figure}
Uncomment the code on this slide to include your own image from the same directory as the template .TeX file.
%\begin{figure}
%\includegraphics[width=0.8\linewidth]{test}
%\end{figure}
\end{frame}

%------------------------------------------------

\begin{frame}[fragile] % Need to use the fragile option when verbatim is used in the slide
\frametitle{Citation}
An example of the \verb|\cite| command to cite within the presentation:\\~

This statement requires citation \cite{p1}.
\end{frame}

%------------------------------------------------

\begin{frame}
\frametitle{References}
\footnotesize{
\begin{thebibliography}{99} % Beamer does not support BibTeX so references must be inserted manually as below
\bibitem[Smith, 2012]{p1} John Smith (2012)
\newblock Title of the publication
\newblock \emph{Journal Name} 12(3), 45 -- 678.
\end{thebibliography}
}
\end{frame}

%------------------------------------------------

\begin{frame}
\Huge{\centerline{The End}}
\medskip
\Huge{\centerline{Thank You}}
\end{frame}

%----------------------------------------------------------------------------------------

\end{document} 